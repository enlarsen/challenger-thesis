\chapter{History}\label{history}

This chapter discusses the space shuttle program in order to
provide an overview of the information the security markets might
have incorporated into prices of the shuttle firms.
The chapter begins with an overview of the shuttle program
and how the current design came about.
It is followed by a section on mission 51-L, the {\em
Challenger} mission that was lost.
Following this is a very detailed discussion of the accident
itself to illustrate what was eventually determined to have
happened.
The purpose of this section is to show that the true events
could not have been discerned on the day of the accident.
In fact, the accident appeared to be caused by many things
other than what really caused it.
The sources of incorrect information about the cause of the
accident are covered.
Finally, the chapter closes with a discussion of what
investigators determined to be the actual cause of the
accident and how widespread knowledge of this possible
failure path was known.

\section{Space Shuttle}

The space shuttle program was initiated in the 1960s to
provide relatively inexpensive access to space.
The giant Apollo program was winding up, and the war in Viet
Nam required an increasing portion of the budget.
NASA needed to find a cheaper means of transportation to
space when they devised the Space Transport System or STS.
The lower costs came about from complete reusability of the
components, contrasted with the Apollo program that required
giant disposable rockets.
Complete reusability would make space travel more
economical, and, it was hoped, would allow the program to
pay for itself.\footnote{NASA envisioned selling cargo space
on the STS for satellites, military and commercial.}
A problem remained, however; the program needed
Congressional approval.
If the system were entirely reusable, it would entail
development costs (1970) of \$10 to \$12 billion (spread
over seven or eight years) \cite[p. 57]{lewis}. Congress and
the Office of Management and Budget refused to go along with
this plan, forcing NASA to redesign the system.

In order to receive Congress's approval, NASA shifted the
costs from development to operation.
The craft would be cheaper to develop, but would cost a lot
more to operate.
In doing so, NASA scrapped the idea of a manned first stage
that would carry the orbiter (riding piggy back) to orbit
and then return to earth.
The manned first stage was replaced by a combination of
solid rocket boosters and a liquid tank that would contain
fuel for the orbiter's main engines (located at the aft area
of the orbiter and fed by a series of pipes from the
external tank).
They also scrapped the idea of jet engines in the
orbiter,\footnote{Air-burning jet engines would allow the
orbiter more flexibility in returning to earth or aborting a
mission but would add a lot of weight and design effort.}
complete reusability, and an orbiter with a large wingspan.

These changes put the final development costs at \$6.2
billion in 1972 dollars \cite[vol. 1, p. 4]{rogers}.
The final system, the one in use today, consists of a
delta-winged orbiter mounted on a large, rust-colored tank.
Mounted to the tank are two solid rocket boosters, the
largest ever made.
The external tank that the orbiter is attached to contains
three smaller tanks.
The first, located at the top of the tank, contains liquid
oxygen, some 143,000 gallons at -297 degrees Fahrenheit.
The other tank, located in the aft, or tail, section of the
external tank contains liquid hydrogen at -423 degrees
Fahrenheit \cite[vol. 1, p. 8]{rogers}.
The third, called the intertank, sits
between the liquid oxygen and hydrogen tanks and houses
instrumentation needed to assure operation of the external
tank.
These three tanks comprise the external tank, which is
manufactured by Martin Marietta and is 154 feet long and
27$\frac{1}{2}$ feet in diameter \cite[vol. 1, p.
8]{rogers}.

The main shuttle engines feed from this external tank
through a maze of pipes, pumps, and valves.  The engines
fire for about 8$\frac{1}{2}$ minutes after liftoff and can
generate
375,000 pounds of thrust (at sea level) at 100 percent
throttle, burning a mixture of liquid hydrogen and liquid
oxygen supplied from the external tank \cite[vol. 1, p.
7]{rogers}.
The main engine's range of operations extends from 65
percent to 104 percent of throttle \cite[vol. 1, p. 8]{rogers}.
The engines themselves
are manufactured by Rocketdyne, a subsidiary of Rockwell
International, the manufacturer of the orbiter.

Attached to the external tank are two of the largest solid
rocket boosters ever manufactured.
They provide about 80 percent of the total thrust and burn
for roughly two minutes until exhausted.
The boosters themselves are made up of four segments that
are shipped to NASA by rail.
The segmented design allows for easier shipment to and from
the Morton Thiokol's (the manufacturer's) plant in Utah to NASA.
The SRBs, as the Solid Rocket Boosters are called, are 116
feet long when the four segments are joined and held
together with 177 steel pins.
The joint (called a field joint) itself is made up of a tang
and clevis;  a clevis is a U-shaped part into which a tang
fits.
The joint is depicted in Figure~\ref{fieldj}.
\begin{figure}[p]
\vspace{4.75in}
\caption{The solid rocket booster field joint
\protect\cite[vol. 1, p. 57]{rogers}.}
\label{fieldj}
\end{figure}

The gap between the tang and clevis is sealed with a pair of
rubber rings called O-rings.
The segmented design represents the weakness in the 
solid rocket boosters.
By joining the boosters together in four segments, four
escape paths for the hot gasses are provided.
The O-rings are supposed to provide a sealing action to
prevent these hot gasses from escaping through a field
joint, but the O-rings were never meant to survive the heat
of the gasses themselves.
Zinc chromate putty is packed between the solid rocket core
and the O-rings.
The putty is flame-proof, and, in theory, should become
plastic and melt around the O-rings, protecting them from
the heat.


This current design represents a compromise over
the original design.  Higher operating costs (not complete
reusability like the original design) were traded off for
much lower development costs (half of the original amount).
The compromise represented a compromise in crew safety as
well.  Although the SRBs provided more thrust than an
equally sized liquid propellant system, they exposed the
crew to a two minute window when escape was
``nonsurvivable'' \cite[p. 59]{lewis}.
Yet solid rocket technology was considered one of the
most reliable around.  In fact, ``before the Challenger
accident, it was believed by some shuttle observers\ldots
that the SRBs were as safe and reliable as rockets could
be'' \cite[pp. 63--64]{lewis}.  Also, ``the perception that
the SRBs were virtually foolproof was widely shared in NASA
during shuttle development and was strongly communicated to
the news media correspondents'' \cite[p. 64]{lewis}.

\section{Mission 51-L}

Flight 51-L was the {\em Challenger} mission that was lost. 
NASA's numbering system designated this mission 51-L with the
first digit (``5'') representing the fiscal year that the
launch was originally supposed to occur.  The second digit
indicated where the launch was to occur (here ``1''
represents Kennedy Center).  The third, a letter, specifies
which launch in that fiscal year (with ``A'' as the first
mission).  Mission 51-L was unique
because Christa McAuliffe, the first ``Teacher In
Space,'' was aboard.  She and her six other
crewmates\footnote{They were Commander Francis R. (Dick) Scobee,
Pilot
Michael John Smith, Mission Specialist One Ellison S.
Onizuka, Mission Specialist Two Judith Arlene Resnik,
Mission Specialist Three Ronald Erwin McNair, and Payload
Specialist Two Gregory Bruce Jarvis.  Christa McAuliffe's
title was Payload Specialist One.} faced four postponements
and one scrub of the mission.  The flight was originally
scheduled for July of 1985 \cite[vol. 1, p. 10]{rogers}, but
changes in payloads pushed the date back to November, 1985. 
More delays occurred, and the mission was moved to January of
1986.  As the mission was pushed back again and again, there
was increasing pressure to launch to keep up with NASA's
ambitious launch schedule.  NASA also faced a constraint in
that a payload needed to be launched by March to investigate
Halley's Comet; a {\em Challenger} launch delay could
prevent this comet probe from being launched in March.

The launch was finally scheduled for Tuesday, January 28,
1986.  The temperature at the time of launch was 36 degrees
Fahrenheit, causing some concern among the shuttle's
contractors.  Executives of Rockwell International, the builder of
the
orbiter, expressed concern about the ice on the
launch pad.\footnote{The launch structure was covered with
ice because, fearing frozen pipes, NASA officials decided to
allow water to trickle from the water pipes.}  The concern
centered around the possibility that ice might break off
under the enormous acoustic shocks when the engines are
ignited.  Falling ice could dislodge the heat
resistant tiles that line the underside of the shuttle.
The tiles protect the orbiter from the heat of re-entry.

NASA had sent an ``ice team'' to investigate the problem. 
The ice team measured the temperature of the right solid
rocket booster and found it to be about 19 degrees
Fahrenheit, well below the manufacturer's (Morton Thiokol's)
contracted low temperature of 40 degrees Fahrenheit.  
The contract between NASA and Thiokol stipulated that the
SRBs operate above 40 degrees Fahrenheit.  The temperature
finding was not reported up the chain of command.

Morton Thiokol's engineers\footnote{Morton Thiokol
manufactures the
solid rocket boosters.} were concerned as well about the
low temperature.  The engineers had never fired test rockets
at such a low temperature and were concerned about
something they observed in January, 1985.  The January, 1985
launch was unique because it had been the coldest to date. 
Although they had no
statistical data to suggest there was a correlation, the
engineers felt there was a correspondence between
temperature and the O-rings not sealing.  By not sealing,
the O-rings had allowed hot gasses to escape through the
joint, burning (eroding) several hundreths of an inch from the
secondary O-ring.  Damage occurred in both solid rocket
boosters in the January, 1985 launch.

The engineers were overridden, however, and Thiokol gave
approval for the launch.  The engineers were put in the
position of having to prove that the O-rings would fail,
rather than prove that they would work.  There was also no
launch constraint for temperature; no one really knew how
cold it could be.

\section{The Accident}

The Space Shuttle {\em Challenger}, Mission 51-L
of Tuesday, January 28, 1986, had been postponed four
times and scrubbed once. When the craft was finally launched
at 11:38 am EST, the
mission progressed normally until T+73 seconds\footnote{This
notation (T+73 seconds) refers to 73 seconds after launch.}
when the {\em Challenger} exploded and killed all seven crew 
members.  It was not immediately obvious what had happened,
and some of the people on the ground thought the smoke 
indicated the planned-for separation of the orbiter from the
external tank.

In spite of the uncertainty about the cause of the problem,
there was visual indication of some problems immediately
after ignition of the solid rocket boosters; no one noticed
the problem until enhanced photographs of the shuttle were
analyzed.
The right SRB emitted several puffs of black smoke at 0.678
seconds into the launch (T+0.678).
A total of nine puffs of smoke appeared between 0.836 and
2.500 seconds at a lower joint on the right SRB \cite[vol.
1, p. 19]{rogers}.
The smoke's color suggested that the O-rings between joins
had burned, causing what is called ``blow by.''
It appears that no one at NASA noticed this blow by, and had
anyone, it would have been impossible to stop the solid
rockets, as they burn continuously until they are exhausted.
At 2.733 seconds, the last positive evidence of smoke was
visible from the right aft field joint;\footnote{The smoke
first appeared on camera E60, a 16mm motion picture camera
with a 32mm lens.
The camera was located 1270 feet from the shuttle and shot
100 frames per second \cite[vol. 3, p. N-9]{rogers}.} the
shuttle was moving upward and the aft smoke intermingled
with plumes from the rocket.

All this time the shuttle's main engines had been burning
(they were ignited 6.6 seconds before launch in order to get
up to full thrust by the time of launch).  The shuttle
system was bolted to the launch structure, which bends under
the force of the main engines.  The launch structure bends
back in what is called a ``twang'' motion.  At this time the
pyrotechnic bolts are exploded, releasing the launch
structure's stored energy.  The Roger's Commission report
explains, ``the maximum structural loads on the aft field
joints [where the black smoke appeared] of the Solid Rocket
Boosters occur during the `twang,' {\em exceeding even
those of the maximum dynamic pressure period experienced
later in flight} [emphasis added]'' \cite[p. 19]{rogers}.
This pressure is part of the reason the joint failed.

The command to increase thrust of the main engines to 104
percent of their rated maximum was given at 4.339 seconds
into the mission.  In Houston, Stephen Nesbitt, public
affairs officer for the Johnson Space Center Mission Control
announced the mission progress,
\begin{singlespace}
\begin{quotation}
\noindent
\ldots roll program confirmed.  {\em Challenger} now heading
down range.  Engines beginning to throttle down to 94
percent.  Normal throttle for most of the flight is 104
percent.  Will throttle down to 65 percent shortly.  Three
engines running normally\ldots
Velocity 2,257 miles per hour.  Altitude 4.3
nautical miles.  Engines throttling up.  Three engines now
at 104 percent.
\end{quotation}
\end{singlespace}
The roll program meant the shuttle rolled over on its back,
moving the leaking right SRB away from any news cameras.  No
one (outside of NASA) could then visually determine what had
happened, since
the footage showed the completely normal left solid rocket
booster.  On Saturday, February 1, 1986 (the accident
occurred on Tuesday), the {\em
New York Times} reported that ``the television pictures of
the {\em Challenger} explosion seen by the public show the
right booster on the far side of the camera, behind the
shuttle, which had rolled on its back.  NASA is examining
photographs taken from other angles, and it was reported
that one view suggested a jet of flame shooting from the
side of the right booster rocket.''

Telemetry\footnote{Telemetry consists of readings taken from
various sensors and instruments aboard the shuttle and
radioed back to NASA.
Most of the data is simply stored on computer tape for later
analysis with little actually presented in real time to the
controllers at Marshall and Kennedy.} began recording an
anomaly in the right SRB at 5.674 seconds; the pressure was
11.8 psi above nominal.
Next at T+7.724 the shuttle began a programmed roll, rolling
the craft on its back.
The engines were then throttled back to 94 percent of rated
power.
Everything appeared normal as the shuttle entered,
between 37 and 64 seconds into the mission, an area of high
altitude wind shear, placing wildly varying forces on the
craft.
The computers automatically adjusted, executing pitch and
yaw maneuvers to adjust the crafts attitude.

At 51.860 seconds into the mission the main engines again
throttled up to 104 percent.
Mission Control, Houston, said, ``go at throttle up.''
Commander Scobee replied, ``roger, go at throttle up,''
acknowledging the command as the computers executed it.
The first evidence of a flame appeared at 58.788 seconds.
It was localized to the right hand solid rocket booster, in
computer-enhanced film of the launch.
Camera E207 showed the flame grow larger, becoming a
well-defined plume one frame later at 59.262 seconds.
A pressure differential at T+60 seconds between the right
hand and left hand solid rocket boosters was reported by
telemetry (the right booster's pressure was lower).

The plume began to be deflected at 60.238 seconds by
aerodynamic forces toward the liquid oxygen and hydrogen
filled external tank.
The shuttle's computers began to compensate for the thrust
differential between the two solid rocket boosters.
Telemetry reported the onboard computers automatically
directed the left solid rocket booster's thrust vector
control (nozzle) to correct for the increased yaw introduced
by the malfunctioning right hand booster.
For about nine seconds the control computers continued to
correct for the increasing yaw and pitch rate errors
introduced by the damaged right hand booster with little
success.

Aerodynamic forces were beginning to tear the craft apart,
and the control systems continued in vain to correct for
them.
Then, at 64.660 seconds the external tank began leaking
liquid hydrogen as indicated by telemetry.
At 72.204 seconds the right hand and left hand SRBs
experienced massively divergent yaw rates, then pitch rates.
This was caused when the lower strut that attached the right
hand SRB to the external tank was torn away allowing the
right hand SRB to rotate freely.
The yaw and pitch introduced was outside the range of
possible compensation by the on-board control computers.

The tearing off of the strut released about 2.8 million
pounds of thrust as massive amounts of hydrogen poured
through the tear in the external tank.  The right SRB swung
around and impacted the liquid oxygen tank causing total
failure of the external tank at 73.137 seconds.
The liquid hydrogen and oxygen spewing from the failed tank
ignited a few milliseconds later into a huge fireball,
enveloping the {\em Challenger} and causing massive
structural failure of the orbiter.  Film footage indicated
the orbiter broke into several sections, one of which was
the crew compartment.

Autopsy reports indicated that at least some of the crew
survived the explosion, as they activated their emergency
oxygen tanks, and the {\em g} forces were well within human
survivability.
The crew compartment reached a height of 65,000 feet (12
miles) before it began its descent, striking the ocean 165
seconds later at a speed of 204 miles per hour \cite[p.
177]{lewis}.
The 200 {\em g} force experienced on ocean impact was
outside human survivability limits.\footnote{Dr. Joseph
Kerwin of Life Sciences at the Johnson Space Center found
evidence that the crew survived the explosion, but died
either from ocean impact or the sudden loss of air after the
compartment was thrown free of the explosion.}

Stephen Nesbitt of Johnson continued his commentary,
``Flight controllers are looking very carefully at the
situation.
Obviously a major malfunction.
We have no down link.
We have a report from the flight dynamics officer that the
vehicle has exploded.''
There was thus no real understanding of what had happened or
why.
On the surface it appeared as though the craft exploded
because the throttle to the main engines was increased to
104 percent.
The explosion occurred approximately three seconds after
Commander Scobee relayed  ``go at throttle up.'' to mission
control.

\section{Cause of the Accident}

President Reagan appointed the Roger's Commission (called
the Presidential Commission) to investigate the accident. 
The Commission was chaired by former Secretary of State
William Rogers.
The Roger's Commission investigated each possible cause of
the accident,  dividing the possible causes into the
following areas:
\begin{singlespace}
\begin{enumerate}
\item external tank,
\item space shuttle main engines,
\item orbiter and related equipment,
\item payload/Orbiter interfaces,
\item payload, inertial upper stage, and support equipment,
\item solid rocket booster.
\end{enumerate}
\end{singlespace}
Here we will only discuss possible failure of the external
tank, orbiter motors, and solid rocket boosters.

\subsection{External Tank}

The large, rust-colored external tank contains an oxygen
tank, a hydrogen tank and an intertank between the two.
The tank is manufactured by Martin Marietta (main
contractor).
Visibly, this tank exploded, suggesting to observers that
this was the cause of the explosion.
In fact, the right SRB, after tearing free of its aft
connection to the external tank began rotating.
This allowed it to rotate around and strike the liquid
hydrogen tank, releasing great quantities of hydrogen which
subsequently ignited.
It would have been almost impossible to discern these events
at the time of the accident as they were only obtained from
telemetry analysis and enhanced photographs.
Recall that the shuttle had seconds earlier performed a roll
maneuver that placed the right SRB away from the cameras of
the news media.

The Roger's Commission investigated several possible
failures with the external tank, rejecting all of them.
The first cause investigated was that there was a premature
detonation of the range safety destruct
devices.\footnote{These explosive devices are used to
remotely destruct the external tank if it is headed toward a
populated area.}
The destruct devices were recovered from the ocean and found
intact and therefore could not have prematurely exploded.

The next cause investigated was a structural failure of the
tank.
Some small imperfection in the tank could have grown in size
due to mission stresses and resulted in total collapse.
Upon analysis of the construction history test data and
x-rays, this possibility was rejected.

Damage to the hydrogen tank at liftoff was considered and
rejected after detailed analysis of the photographic
evidence showed no vapor or frost to indicate a leak.
This possible cause was thus rejected.

Next the commission turned to an analysis of structural
loads on the external tank and whether the rated loads had
been exceeded.
The commission found that there had been no excessive loads
on the external tank up to the explosion.

Overheating of the external tank was another possible cause
that was analyzed.
After analyzing the data, overheating of the external tank
was ruled out as a possible cause of the accident.

The commission thus ``\ldots found nothing relating to the
External Tank that caused or contributed to the cause of the
accident'' \cite[vol. 1, p. 42]{rogers}. Speculation in the
press did center on the external tank, largely because it
contained highly explosive hydrogen.
The morning after the accident, the {\em New York Times}
reported that the external tank had been the cause of the
accident \cite[p. A1]{nytexternal}, ``\ldots suspicions
quickly focused on the craft's huge external fuel tank, a
potential bomb that carried more than 385,000 gallons of
liquid hydrogen and more than 140,000 gallons of liquid
oxygen at liftoff.''

\subsection{Space Shuttle Main Engines}

The space shuttle's three main orbiter engines were built by
Rocketdyne, a division of Rockwell International.
The engines had been the subject of much
controversy during their development as there were a number
of test failures \cite{lewis}.
This contrasted with the mostly trouble-free development of
the SRBs.

Additionally, the casual observer would have guessed the
main engines failed as the shuttle exploded approximately
three seconds after Commander Scobee said, ``roger, go at
throttle up.''
According to a GAO report, avoiding this throttle up
would make the launch safer, ``\ldots eliminating or
reducing about 175 potential failure modes, according to
NASA''
\cite{gao89}.  The day after the accident, the {\em New York
Times} reported, ``certainly, the external fuel tank was
being subjected to great mechanical stresses at the instant
it blew up; the explosion occurred a few seconds after the
shuttle's main engines were boosted to full power''
\cite[p. A4]{nytexternal}.

However, the commission, after a thorough investigation of
the main engines, found that the main engines had not failed.
The only failure they found was caused when the computers
shut down the main engines because the oxygen to hydrogren
ratio increased, which caused the main engines to overheat.
This was because the hydrogen they normally burn was
escaping through a hole in the external tank caused by the
right SRB tearing its strut off the external tank.
The fuel burned by the main engines was liquid hydrogen, with
liquid oxygen provided to enable the burning of hydrogen to take
place in the upper atmosphere where there is very little
oxygen.

\subsection{Solid Rocket Boosters}

Both SRBs had been manually exploded by the range safety officer
after the left SRB appeared headed toward New Smyrna Beach,
Florida.
Operating the range safety devices required that both SRBs
be destroyed, and since the charges detonated at 110 seconds
into the launch (the craft exploded 73 seconds into the
launch), they could not have caused the accident.
After an extensive analysis, the commission found that ``the
left Solid Rocket Booster, and all components of the right
Solid Rocket Booster, except the right Solid Rocket
Motor,\footnote{The Solid Rocket Motor is simply the solid
rocket booster without a nozzle and nose cone.}
did not contribute to or cause the accident''
\cite[vol 1, p. 53]{rogers}.  The {\em New York Times} of
January 29, 1986 (the day after the accident) questioned:
``Might one of the joints between sections of a booster rocket have failed,
somehow forcing a jet of white hot gas through the thin skin
of the external fuel tank?'' \cite{nytexternal}

The Commission began to concentrate on the solid rocket
motor.
After eliminating other causes, the commission studied the
joints between the four segments of the SRB, and found that
the aft field joint had failed allowing hot gasses to breech
the SRB and ignite the liquid hydrogen as it spewed from the
damaged external tank.
The Commission ruled that the field joint had suffered from
a faulty design and that the O-rings in the joint had failed
to seal.
The Commission did not merely find that the O-rings failed,
but that the problem was known by engineers at both NASA and
Morton Thiokol (the designer of the field joint) and had not
been addressed.

\subsection{History of O-Ring Problems}

The O-ring problem dated back to the early 1980s, but it was
only in 1985 that the problem became acute, resulting from
the method Thiokol and NASA chose to test whether the
O-rings were sealed.
This test involved forcing air into the joint from a test
port to seal the secondary O-ring.
NASA originally tried 50 psi of pressure, but found that
inadequate to seal the O-rings.
They continued to increase the pressure of the air forcing
into the joint until they were using 200 psi.
There was a major problem with using 200 psi of pressure;
the zinc chromate putty that protected the O-rings from hot
gasses often developed holes that allowed jets of hot gasses to
reach the O-rings.
This resulted in O-ring erosion.  Normally, the putty
becomes plastic and flows around the O-rings, but holes
blown in the putty focus hot gasses on
the O-rings, resulting in erosion.

Each of the four joints in the SRB contained two O-rings.
The primary O-ring was designed to provide the sealing
necessary for the joint.
The other O-ring, the secondary O-ring, was redundant and
allowed the joint to be considered fail safe.
Nevertheless, on November 24, 1980 in what is called a
``Critical Items List,'' the O-rings were classified
``Critical 1R'' whose failure would result in ``loss of
mission, vehicle, and crew due to metal erosion,
burnthrough, and probable case burst resulting in fire and
deflagration'' \cite[vol. 1, p. 239]{rogers}.

The problem with the O-rings surfaced in January, 1985 when
flight 51-C's field joints experienced blow-by and scorching
(erosion) of the secondary O-ring.
In April of 1985, flight 51-B experienced primary O-ring
erosion in the right and left boosters.
The mounting evidence of the O-ring problem and its
correlation with lower launch temperatures caused the
engineers at Morton Thiokol to oppose the launch of 51-L
(the Challenger mission that was lost).
The main reason was the unseasonable cold; it was predicted
to drop to 18 degrees the night of January 27, 1986 (the
shuttle was launched January 28).
The engineers believed that there was a correlation between
temperature and O-ring erosion, but they had no statistical
data or burn tests at that temperature to prove it.
There was no launch constraint\footnote{A launch constraint
would prevent launch should the condition described in the
constraint exist at the time of the launch.
For instance, rain was a launch constraint because it would
damage the heat resistant tiles on the underside of the
shuttle.} for low temperature.
The only constraint from Morton Thiokol's point of view was
that the SRBs not be launched under 40 degrees~F.
\footnote{Morton Thiokol specified that the lowest
operating temperature for the SRBs was 40 degrees
Fahrenheit.}
The ambient temperature at launch was 36 degrees~F.

Because of the cold temperatures, NASA began running water
through the pipes of the launch structure to prevent the
pipes from freezing.
The running water soon formed ice.
In order to assess the effect of this ice on the launch
structure, NASA sent out an ``ice team.''
They found that the right SRBs aft section measured about 19
degrees~F, much colder than it should have been at launch.
This finding was not reported up the command
hierarchy.
It is not known why they did not report this finding up the
chain of command; speculation is that the joint was so cold
that the reading was disregarded as an instrument
malfunction.
Later, Richard Feynman, Nobel Prize winning physicist from
Caltech and a Presidential Commission member, found that the
instrument was properly calibrated and the low reading was
probably the result of
winds that had blown across the
external tank (which was filled with liquid hydrogen and
oxygen hundreds of degrees below zero) during the night and onto the right
SRB.

The launch was approved, however, over the heads of the
engineers.
A Morton Thiokol manager signed a paper that allowed the
launch to proceed.
The CEO of Morton Thiokol, Charles S. Locke, stated after
the accident (in a March 16, 1986 {\em Business Week}
interview), ``If we'd been consulted here [at Morton Thiokol
headquarters in Chicago], we'd never have
given clearance, because the temperature was not within the
contracted specs'' \cite[p. 82]{bw}.


The concerns about temperature were never forwarded to
managers higher up at NASA or Morton Thiokol.
In the end, ``the Commission concluded that the Thiokol
Management reversed its position and recommended the launch
of 51-L, at the urging of Marshall [Space Center] and
contrary to the views of its engineers in order to
accommodate a major customer'' \cite[vol. 1, p.
104]{rogers}. One reason, perhaps, for this overruling of the
engineers is that ``[on] January [21], 1986 NASA announced
another plan to develop a second source for shuttle [solid]
motor production'' \cite{gao86,gao89}.
In other words, a few days before the launch, NASA announced
that it would be seeking another source for the SRBs, quite
an incentive to go along with NASA's desire to launch.

\subsubsection{Cook Memos}

Richard C. Cook, a budget analyst with NASA, was assigned
the task of identifying ``threats'' to the budget.
A threat was anything that could impact the budget in a
major way.
One threat he found was the O-rings.
The memo, which he directed to his superior, Michael B.
Mann, was dated July 23, 1985 (six months before the
accident) and begins, ``earlier this week you asked me to
investigate reported problems with the charring of seals
between SRB motor segments during flight operations.
Discussions with program engineers show this to be a
potentially major problem affecting both flight safety and
program costs'' \cite[vol. 4, p. 391]{rogers}.

On Monday, February 3, 1986, about one week after the accident, and the
day President Reagan issued Executive Order 12546, founding
the Presidential Commission on the Space Shuttle Challenger
Accident, Cook wrote another memo at the request of his
superior, Michael B. Mann.  In this memo, Cook said, 
\begin{singlespace}
\begin{quotation}
\noindent
there is a growing
consensus that the cause of the Challenger explosion was a
burnthrough in a Solid Rocket Booster at or near a field
joint.  It is also the consensus of engineers in the
Propulsion Division, Office of Space Flight, that if such a
burnthrough occurred, it was probably preventable and that
for over a year the Solid Rocket Boosters have been flying
in an unsafe condition.  This has been due to the problem
of O-ring erosion and loss of redundancy caused by unseating
of the secondary O-ring in flight \cite[vol. 4, p.
393]{rogers}.
\end{quotation}
\end{singlespace}
\subsubsection{Other Documents}

The earliest known mention of the O-ring problem occurred in
1977 in a briefing chart by Marshall engineer Leon Ray.
He stated that failing to change the joint and leaving it as is
was ``unacceptable---tang can move outboard [away from the
clevis] and cause
excessive joint clearance resulting in seal leakage''
\cite[vol. 1, p. 233]{rogers}.  In a memo dated January 9,~1978, 
written by Leon Ray, and signed by chief of the Solid
Rocket Motor branch at Marshall, John Q. Miller, Miller
urges that the joint be redesigned to ``prevent hot gas
leaks and resulting catastrophic failure''
\cite[vol. 1, p. 234--235]{rogers}.  Once again, in a memo
written by Leon Ray and signed by John Q. Miller dated
January 19, 1979, the joint design is found inadequate.
Ray states, ``we find the Thiokol position regarding design
adequacy of the clevis joint to be completely
unacceptable\ldots'' \cite[vol. 1, p. 236]{rogers}.

Roger Boisjoly, an engineer with Morton Thiokol, became
concerned about erosion of O-rings after the 1985 flights
(when higher pressures were used to test the sealing of the
O-rings, causing holes to form in the heat-resistant zinc
chromate putty protecting the O-rings).
In a memo to Robert Lund (Vice President of Engineering,
Morton Thiokol)
dated July 31, 1985 (six months before the accident), he
stated, ``this letter is written to insure that management
is fully aware of the seriousness of the current O-Ring
erosion problem in the SRM joints\ldots''  He stated that if
the secondary O-ring failed, and ``\ldots it is a jump ball
as to the success or failure of the joint\ldots,'' ``the
result would be a catastrophe of the highest order---loss of
human life'' \cite[vol. 1, p. 249]{rogers}.

A ``seal task force'' was set up in order to address the
problem, but they made little progress.
In a memo dated October, 3, 1985, Robert Ebeling, manager of
Morton Thiokol's Solid Rocket Motor ignition system, begins,
``HELP! The seal task force is constantly being delayed by every
possible means'' \cite[vol. 1, p. 252]{rogers}. He concludes
with, ``This is a red flag'' \cite[vol. 1, p.
252]{rogers}.

\section{Summary}

This chapter discussed the history of the space shuttle
program and how the present design came about.
The discussion then moved on to show that a wide group of
people had knowledge of a possible failure of the space
shuttle.
This information about the true cause of the shuttle
explosion could not have been readily discerned after
the Challenger explosion because the explosion was caused by
ignition of liquid hydrogen from the external tank, casting
suspicion on the tank itself.
Additionally, the explosion occurred three seconds after
Commander Scobee confirmed, ``go at throttle up.'' A casual
observer would guess the orbiter's engines 
or the external liquid hydrogen/liquid oxygen tank had
failed, not the
solid rocket boosters, which represented mature technology.
Lewis says,
\begin{singlespace}
\begin{quotation}
\noindent
An astonishing aspect of this situation [the O-ring problem]
was that so far as the public was concerned, it was one of
the best-kept secrets of the space age.
The documents describing it were not classified and did not
need to be.
They were buried in the files at NASA headquarters in
Washington and the Marshall Space Flight center in
Huntsville, Alabama.\par
\bigskip
\noindent
Along with the general public, the astronauts who were
flying the shuttle were unaware of the escalating danger of
joint seal failure.  So were the congressional committees
charged with overseeing the shuttle program.\par
\bigskip
\noindent
NASA never told them that the shuttle had a problem.
\end{quotation}
\end{singlespace}

%\footnote{NASA planned to design
%and build a new generation of solid rocket boosters called ASRM
%(Advanced Solid Rocket Motor) that would, due to their increased
%thrust power, eliminate the need to throttle up.  Morton Thiokol
%dropped out of the bidding for these ASRMs because as some
%observers
%believed they could not get it through Congress.  ``NASA
%Administrator
%James C. Fletcher last week denied reports that Thiokol dropped
%out
%because it could not win the competition due to political fallout
%\cite{noasrms}}
