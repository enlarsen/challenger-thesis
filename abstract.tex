\chapter*{Abstract} \addcontentsline{toc}{prelim}{ABSTRACT}

The theory of efficient markets states that capital markets instantly adjust to incorporate all relevant information. When new information arrives about a firm or an industry, efficient markets theory dictates that the price of the firm or industry in question move to incorporate this information. This theory provides a basis for studying the effects of news on firms. By performing what is called an event study, a researcher can assess how the market perceives some news on the company in question.

The study here focuses on the effect of the explosion on Tuesday, January 28, 1986, of the space shuttle {\em Challenger}. The null hypothesis that was disproved was that due to the explosion, firms involved in the shuttle program would expect to see a drop in returns on the day of the accident. This is what was seen, except for one company, Morton Thiokol, the manufacturer of the solid rocket booster. The cause of the accident was completely unknown until a few days later, but the market reacted on the day of the accident; Morton Thiokol's firm value dropped twelve percent. The reaction largely stemmed from knowledge that the joint design of the solid rocket boosters was faulty and nothing had been done about it.

The market expected substantially lower returns in the future for Morton Thiokol. Morton Thiokol dropped out of the bidding for NASA's next generation solid rocket boosters, some believe because of political pressure. The contract had to receive Congress's approval, but after the revelations about the faulty design of the solid rocket boosters joint, many felt it unlikely that Morton Thiokol would ever win the contract. Morton Thiokol also performed {\em pro bono} the design work on the new solid rocket boosters.

On the day of the accident and because of the faulty design of the solid rocket booster's joints, the market indicated that Morton Thiokol would face some future loss. The explosion caused information of the faulty joint to become widely held, and the market reacted to past information it held about substantial penalties paid by at-fault companies.
