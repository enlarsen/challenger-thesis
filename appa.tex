\appendix
\chapter{Performing Event Studies}

This appendix provides two programs for performing event
studies used in this thesis.  The first, called es

\begin{singlespace}
\begin{scriptsize}
\begin{verbatim}
*! version 1.0 06/18/92
program define es
        version 3.0
        local options "BS(int 0) Caldt(real 0) Start(int 0) End(int 0) BE(int 0) AE(int 0)"
        parse "`*'"
        
        capture drop event ashift bshift
        local i=_N
        local found=0
        local jl=0
        local ju=_N+1
        local ascnd=caldt[_N] > caldt[1]
        while (`ju'-`jl' > 1) {
                local jm=int((`ju'+`jl')/2)
                if((`caldt'>=caldt[`jm']) == `ascnd') {
                        local jl=`jm'
                }
                else {
                        local ju=`jm'
                }
        }
        local edate = `jl'
        di "Event date " caldt[`edate']
        local bevent=`edate'-`be'
        local aevent=`edate'+`ae'
        local edelta=`be'+`ae'+1
        di "Event study on " "$S_FN"
        di "Event window from " caldt[`bevent'] " To " caldt[`aevent']
        di "   (" `edelta' " trading days)"
        if (`bs' == 1) {
                di "   (testing for beta shift)"
        }
        if `caldt' == caldt[`edate'] {
                local lower=`edate'-`start'
                local upper=`edate'+`end'
                if(`bs'==1) {
                        quietly {
                                generate bshift=cond(_n>`edate' & _n<=`upper',ewretd,0) in `lower'/`upper'
                                generate ashift=cond(_n>`edate' & _n<=`upper',1,0) in `lower'/`upper'
                        }
                }
                generate event=cond(_n>=`bevent' & _n<=`aevent',1,0)
                di "Event study regression from " caldt[`lower'] " to " caldt[`upper']
                if(`bs' == 1) {
                        regress ret event ewretd ashift bshift in `lower'/`upper'
                }
                else {
                        regress ret event ewretd in `lower'/`upper'
                }
                drop event
                if (`bs' == 1) {
                        drop bshift ashift
                }
        }
        else {
                di "Could not find " `caldt' " in the data"
        }
     
        end
\end{verbatim}
\end{scriptsize}
\end{singlespace}

\begin{singlespace}
\begin{scriptsize}
\begin{verbatim}
//CRSP1 JOB (,,,5),TIME=(1,04)
/*JOBPARM T=1,Q=H
//S1 EXEC SAS606
//R DD DSN=ACS.CRSP.DAILY.RETURNS,DISP=SHR,UNIT=TAPE
//I DD DSN=ACS.CRSP.DAILY.INDEX.SAS,DISP=SHR
//X DD DSN=ELARSEN.MISC.SOURCE(LIST),DISP=SHR
//Y DD DSN=ACS.CRSP.HEADER.SAS,DISP=OLD
//SYSIN  DD *
  options nonotes;

DATA ELIST;
  INFILE X;
  INPUT edate EPERM ELABEL $ 14-78;
  IF EDATE > 620000;

DATA _NULL_;
   SET ELIST END=LAST;
   CALL SYMPUT('MPERM']]LEFT(_N_), TRIM(EPERM));
   CALL SYMPUT('MLABEL']]LEFT(_N_), TRIM(ELABEL));
   CALL SYMPUT('MDATE']]LEFT(_N_), TRIM(EDATE));
   call symput('ldate']]left(_n_),
      put(input(put(edate,z6.),yymmdd6.),weekdate.));
   IF LAST=1 THEN CALL SYMPUT('NEVENTS',_N_);

%MACRO EXPPERMS(NUMBER, VAR);
%LOCAL I;
  &VAR in (
%DO I=1 %TO &NUMBER;
   &&MPERM&I
   %END;
   )
%MEND EXPPERMS;

%MACRO GETCNAME;
%LOCAL I;
%DO I=1 %TO &NEVENTS;
   IF PERM=&&MPERM&I THEN CALL SYMPUT('CNAME']]LEFT(&I), TRIM(NAME));
   %END;
%MEND GETCNAME;


Data tmpcname;
   set y.header;
   if %expperms(&nevents, perm);

proc sort data=tmpcname; by perm decending namedt;
proc sort data=tmpcname nodupkey; by perm;
DATA _NULL_;
   SET tmpcname;
     %GETCNAME;

DATA RET(KEEP=PERM RET T MISSING);

INFILE R MISSOVER;
RETAIN BEGRET ENDRET

INPUT @9 PERM IB4. @13 SEGMENT IB4. @ ;

  IF %EXPPERMS(&NEVENTS, PERM) THEN DO;
IF SEGMENT=1 THEN
  input @97 begret ib4. @101 endret ib4.;

IF SEGMENT = 11 THEN DO;
 FIRST=BEGRET+1;
 LAST=ENDRET-BEGRET+1;
 MISSING=1;
 DO IRET=FIRST TO LAST;
   IF IRET<=0 THEN IRET=1;
   AT=13+IRET*4;T=IRET+BEGRET-1;
   INPUT @AT RET RB4. @;
   IF RET <-9 THEN RET=.;
   IF RET=. THEN MISSING=MISSING+1;
   IF RET ^=. THEN MISSING=1;
   OUTPUT RET;
   END; END;
END;

DATA I; SET I.DATA; T=_N_; KEEP T CALDT EWRETD;

PROC SORT DATA=RET; BY T;
DATA RET; MERGE RET I; BY T; IF RET^=.; DROP T;
PROC SORT DATA=RET; BY PERM CALDT;


%MACRO REGRESS;
%DO I=1 %TO &NEVENTS;
  DATA _NULL_;
    SET RET;
    IF CALDT=&&MDATE&I AND PERM=&&MPERM&I THEN
      CALL SYMPUT('CDATE']]LEFT(&I),_N_);

  %DO J=75 %TO 125 %BY 25;
    %DO K=1 %TO 10;
       %LET UPPER=%EVAL(&K/2);
       %LET LOWER=%EVAL((&K-1)/2);

      DATA REGS;
        SET RET;
        IF PERM=&&MPERM&I;
        IF %EVAL(&&CDATE&I-&J)<=_N_<=%EVAL(&&CDATE&I+&J-1);
        IF(&&CDATE&I-&LOWER)<=_N_<=(&&CDATE&I+&UPPER) THEN
           EVENT = 1; ELSE EVENT = 0;

 PROC REG NOPRINT OUTEST=TMPEST COVOUT;
  MODEL RET=EVENT EWRETD;

 DATA _NULL_;
  SET TMPEST;
  IF _TYPE_='PARMS' THEN
    CALL SYMPUT('EVENTEST', EVENT);
  IF _TYPE_='COV' AND _NAME_='EVENT' THEN
    CALL SYMPUT('STE', SQRT(EVENT));

DATA TMPRESLT;
   WINDOW=&K;
   TRADE=2*&J;
   ESTIM=&EVENTEST;
  T=&EVENTEST/&STE;
   P=(1-PROBT(ABS(T),(&J*2)-3))*2;

   IF P <= 0.01 THEN SIG="***";
   ELSE IF P <= 0.05 THEN SIG="**";
   ELSE IF P <= 0.1 THEN SIG="*";

PROC APPEND BASE=RESULTS DATA=TMPRESLT; RUN;

%END;
  %END;

PROC PRINT DATA=RESULTS;
  TITLE1 "EVENT STUDY FOR &&CNAME&I Perm: &&MPERM&I";
  TITLE2  "EVENT: &&MLABEL&I";
  TITLE3 "EVENT DATE: &&ldate&I";
  FOOTNOTE1 "* P<=.10  ** P<=.05  *** P<=.01";

  PROC DATASETS LIBRARY=WORK;
     DELETE RESULTS;
 %END;
 %MEND REGRESS;

%REGRESS;
\end{verbatim}
\end{scriptsize}
\end{singlespace}
