\addtocontents{toc}{\protect\noindent CHAPTER} \pagenumbering{arabic} \chapter{Introduction}

This thesis describes the Space Shuttle {\em Challenger} accident on Tuesday, January 28, 1986. The following discussion uses a financial markets analysis to show that the market implicated Morton Thiokol in spite of a number of false signals on the day of the accident. The markets were efficient in determining who was at fault and in ignoring the false signals.

{\em Challenger} exploded 73 seconds after lift-off, killing all seven crew members. On the day of the accident, an observer on the ground would likely have guessed that the accident had been caused by the external tank or the space shuttle main engines, not the solid rocket boosters. However, several days later, the solid rocket boosters were implicated as the cause of the accident.

The stock market was not so easily fooled by appearances. On the day of the accident, the manufacturer of the solid rocket boosters, Morton Thiokol, experienced a statistically significant twelve percent decline in returns. The other space shuttle contractors studied here (Rockwell International, Lockheed, Grumman, and Martin Marietta) experienced much smaller drops in returns, around four percent. This indicates that the market had information about the failure of the Morton Thiokol solid rocket booster.

Chapter~\ref{litrev} discusses relevant financial literature in order to construct a framework for conducting event studies. Event studies allow us to study the returns of the firms in question to assess the impact of the shuttle disaster. The first topic discussed in this chapter is efficient markets literature along with theories arguing that stock prices move randomly, or according to a random walk. The theory of random walks was expanded by Samuelson and Mandelbrot to illustrate that we would expect prices to move randomly in efficient markets. Fama built upon this foundation, devising three tests of market efficiency. Fama pointed out that the interesting result is not whether markets are efficient, but instead what degree of efficiency these markets exhibit. Part of the reason for this is the joint-hypothesis problem which requires that when testing for efficient markets, the researcher must use a model of asset pricing to devise a world where there is no excess demand. Tests of efficiency are confounded by this additional requirement of an asset pricing model; efficiency cannot be proven. The degree of market efficiency becomes the relevant area of emphasis.

Following the discussion of efficient markets is a section on the capital asset pricing model (CAPM), the asset pricing model used in this paper. CAPM is based on the trade-off between return and risk in the capital markets. It is assumed that an investor can obtain what is called the risk-free rate as a return on his investment. The linear relationship between the risk-free rate and the set of minimum variance assets is called the security market line and reflects the cost of risk. This linear relationship is used to devise the market model, which shows that the riskiness of a security or portfolio in question is a linear function of the market portfolio.

A statistical means of testing for abnormal returns can be built from this foundation. This means of testing is called and event study. Abnormal returns are the residuals from an ordinary least squares regression of the security's return on the market portfolio. An event study is study of these residuals to determine what effect an piece of news had on a firm.

The next chapter, Chapter~\ref{history}, discusses the space shuttle program. The various budgetary tradeoffs are discussed as well as the reliability of the two forms of putting the orbiter into space. One method was to use liquid hydrogen and a fuel. The other was to use solid rocket boosters. NASA chose to use a combination of the two. Using both methods, instead of just liquid hydrogen, reduced the development costs of the shuttle.

The development of the liquid propulsion system was overshadowed by test failures and delays. The solid rocket propulsion system, on the other hand, went relatively smoothly. This also compared favorably with the high reliability of the solid rocket technology. Although these would be the largest solid rockets ever manufactured, many in NASA believed that they represented proven, safe technology.

The chapter continues to discuss in detail the accident itself and how this detail was not available on the day of the accident because most information had been recovered from telemetry that had fed straight from the shuttle into NASA's computers. There were visual signs of a problem as well, but the shuttle had executed a roll maneuver and moved the faulty right solid rocket booster away from news cameras. It was only when NASA analyzed footage from other cameras that they found the obvious evidence of a problem with the right solid rocket booster.

An observer on the ground would not have suspected the right solid rocket booster was the culprit as the explosion signaled the large external tank as the cause. Another possible cause was the main engines, a conclusion drawn largely from the fact that the shuttle exploded three seconds after the main engines increased power to 104 percent of the rated maximum.

The chapter continues by describing the possible causes of the accident investigated by the Roger's Commission. After careful analysis by the commission, the right solid rocket booster (SRB) was found to be involved. The Commission eventually concluded that the rubber O-rings that provide sealing in the SRB failed, allowing hot gasses to burn through the solid rocket booster and ignite the external tank. The Commission also found that although the problem with the O-rings was known for some time, nothing was done.

The final section details the history of O-ring problems and how widespread the knowledge was. The reports of problems date back to 1977, but the problem started to become acute in 1985. NASA had been using a technique to assess whether the O-rings were seated. The technique often resulted in hot gasses scorching the O-rings after the solid rocket booster was fired. Morton Thiokol and NASA set up task forces to study the problem, but no progress was made.

Chapter~\ref{empiric} analyzes the empirical results of the event studies performed for the accident. If markets are efficient, it is expected that the market should reflect the real cause of the accident in the price of the firms working on shuttle projects. Although it might have looked as though the cause of the accident were the main engines or the external tank, the market would reflect the true cause of the accident if such information is available. Had the information about the faulty joints not been known, or had the joint not been faulty, it is expected that there would not have been a firm-specific reaction. But the case studied here shows that from information available, Morton Thiokol was the most likely culprit in the accident.