\chapter{Conclusion}

Chapter~\ref{litrev} reviewed the relevant efficient market,
capital asset pricing, and event study literature in order
to provide a foundation for the study that followed.  The
joint-hypothesis problem was reviewed as were problems in
performing event studies.  The joint-hypothesis problem is
caused by the requirement of an asset-pricing model to
accompany
any tests of market efficiency.  This confounds results
of studies of efficiency in markets because the result could
be due to an inappropriate asset pricing model.  This is
one of the reasons that Fama refers to various tests for
degrees of market efficiency rather than tests only for the
existence of efficiency in security markets.

Chapter~\ref{history} reviews the space shuttle program and
the decision making that went into the design of today's
space shuttle.  The space shuttle program was conceived in
the late 1960s as a means of transporting satellites and
people cheaply into space.  The development costs of the
ideal (ideal in the sense of low operating costs) system
were too high, and Congress and the Office of Management and
Budget balked. This caused NASA to redesign the system to
use, among other things, solid rocket technology.  Although
solid rockets had never before been used in manned space
flight, they had a proven record in missiles and satellite
launches.  This introduced a concern into the program
because the crew would be unable to escape the shuttle
during the two minute period when the solid rocket boosters
burn.

The chapter then turned to the accident and analyzed the
accident from the viewpoint of telemetry and enhanced visual
images.  This showed that the right solid rocket booster 
caused the accident, but this was not obvious on the
day of the accident.  In fact, on the day of the accident it
appeared that the external tank and the main orbiter engines
were the cause of the accident.  This was later ruled out by
the Roger's Commission, but the information about suspected
causes leaked out earlier after some of the enhanced images
were released by NASA.

The Commission eventually found that the joint on the solid
rocket boosters was faulty, and that this problem was known for
some time.  The remainder of the chapter details the length
of time this knowledge was known and by whom.  The problem
surfaced in about 1977 during the pre-launch days of the
shuttle.  It was in 1985 that the problem became more
evident because of a method NASA used to assure that the O-rings
between joints of the solid rocket boosters were sealed. 
The test they used blew holes in the putty that guarded the
O-rings against the hot gasses of combustion as the central
core of the SRB burned.  The blow holes allowed hot gasses
to reach the O-rings and burn (or erode) them.

Chapter~\ref{empiric} discussed the statistical results of
this study, finding that the market has quite a depth of
information available about causes.  The results showed what
was expected; all firms experienced negative returns.  The
finding that lends credence to the hypothesis of efficient
markets is that the return on Morton Thiokol was negative
twelve percent on the day of the accident when no one really
knew what caused the accident, but many suspected the
O-rings because it had been a problem for some time.
The market thus performed to expectations by instantly
incorporating information about the cause of the {\em
Challenger} accident into the firm at fault.